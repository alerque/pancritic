\documentclass[]{article}
\usepackage[]{lmodern,color,soul}
\usepackage{amssymb,amsmath,}
\usepackage{ifxetex,ifluatex}
\usepackage{fixltx2e} % provides \textsubscript
\ifnum 0\ifxetex 1\fi\ifluatex 1\fi=0 % if pdftex
  \usepackage[T1]{fontenc}
  \usepackage[utf8]{inputenc}
\else % if luatex or xelatex
  \ifxetex
    \usepackage{mathspec}
  \else
    \usepackage{fontspec}
  \fi
  \defaultfontfeatures{Ligatures=TeX,Scale=MatchLowercase}
\fi
% use upquote if available, for straight quotes in verbatim environments
\IfFileExists{upquote.sty}{\usepackage{upquote}}{}
% use microtype if available
\IfFileExists{microtype.sty}{%
\usepackage{microtype}
\UseMicrotypeSet[protrusion]{basicmath} % disable protrusion for tt fonts
}{}
\usepackage{hyperref}
\hypersetup{unicode=true,
            pdftitle={Using CriticMarkup with pandoc},
            pdfauthor={Kolen Cheung},
            pdfborder={0 0 0},
            breaklinks=true,
}
\urlstyle{same}  % don't use monospace font for urls
\usepackage{color}
\usepackage{fancyvrb}
\newcommand{\VerbBar}{|}
\newcommand{\VERB}{\Verb[commandchars=\\\{\}]}
\DefineVerbatimEnvironment{Highlighting}{Verbatim}{commandchars=\\\{\}}
% Add ',fontsize=\small' for more characters per line
\newenvironment{Shaded}{}{}
\newcommand{\KeywordTok}[1]{\textcolor[rgb]{0.00,0.44,0.13}{\textbf{{#1}}}}
\newcommand{\DataTypeTok}[1]{\textcolor[rgb]{0.56,0.13,0.00}{{#1}}}
\newcommand{\DecValTok}[1]{\textcolor[rgb]{0.25,0.63,0.44}{{#1}}}
\newcommand{\BaseNTok}[1]{\textcolor[rgb]{0.25,0.63,0.44}{{#1}}}
\newcommand{\FloatTok}[1]{\textcolor[rgb]{0.25,0.63,0.44}{{#1}}}
\newcommand{\ConstantTok}[1]{\textcolor[rgb]{0.53,0.00,0.00}{{#1}}}
\newcommand{\CharTok}[1]{\textcolor[rgb]{0.25,0.44,0.63}{{#1}}}
\newcommand{\SpecialCharTok}[1]{\textcolor[rgb]{0.25,0.44,0.63}{{#1}}}
\newcommand{\StringTok}[1]{\textcolor[rgb]{0.25,0.44,0.63}{{#1}}}
\newcommand{\VerbatimStringTok}[1]{\textcolor[rgb]{0.25,0.44,0.63}{{#1}}}
\newcommand{\SpecialStringTok}[1]{\textcolor[rgb]{0.73,0.40,0.53}{{#1}}}
\newcommand{\ImportTok}[1]{{#1}}
\newcommand{\CommentTok}[1]{\textcolor[rgb]{0.38,0.63,0.69}{\textit{{#1}}}}
\newcommand{\DocumentationTok}[1]{\textcolor[rgb]{0.73,0.13,0.13}{\textit{{#1}}}}
\newcommand{\AnnotationTok}[1]{\textcolor[rgb]{0.38,0.63,0.69}{\textbf{\textit{{#1}}}}}
\newcommand{\CommentVarTok}[1]{\textcolor[rgb]{0.38,0.63,0.69}{\textbf{\textit{{#1}}}}}
\newcommand{\OtherTok}[1]{\textcolor[rgb]{0.00,0.44,0.13}{{#1}}}
\newcommand{\FunctionTok}[1]{\textcolor[rgb]{0.02,0.16,0.49}{{#1}}}
\newcommand{\VariableTok}[1]{\textcolor[rgb]{0.10,0.09,0.49}{{#1}}}
\newcommand{\ControlFlowTok}[1]{\textcolor[rgb]{0.00,0.44,0.13}{\textbf{{#1}}}}
\newcommand{\OperatorTok}[1]{\textcolor[rgb]{0.40,0.40,0.40}{{#1}}}
\newcommand{\BuiltInTok}[1]{{#1}}
\newcommand{\ExtensionTok}[1]{{#1}}
\newcommand{\PreprocessorTok}[1]{\textcolor[rgb]{0.74,0.48,0.00}{{#1}}}
\newcommand{\AttributeTok}[1]{\textcolor[rgb]{0.49,0.56,0.16}{{#1}}}
\newcommand{\RegionMarkerTok}[1]{{#1}}
\newcommand{\InformationTok}[1]{\textcolor[rgb]{0.38,0.63,0.69}{\textbf{\textit{{#1}}}}}
\newcommand{\WarningTok}[1]{\textcolor[rgb]{0.38,0.63,0.69}{\textbf{\textit{{#1}}}}}
\newcommand{\AlertTok}[1]{\textcolor[rgb]{1.00,0.00,0.00}{\textbf{{#1}}}}
\newcommand{\ErrorTok}[1]{\textcolor[rgb]{1.00,0.00,0.00}{\textbf{{#1}}}}
\newcommand{\NormalTok}[1]{{#1}}
\usepackage{longtable,booktabs}
\IfFileExists{parskip.sty}{%
\usepackage{parskip}
}{% else
\setlength{\parindent}{0pt}
\setlength{\parskip}{6pt plus 2pt minus 1pt}
}
\setlength{\emergencystretch}{3em}  % prevent overfull lines
\providecommand{\tightlist}{%
  \setlength{\itemsep}{0pt}\setlength{\parskip}{0pt}}
\setcounter{secnumdepth}{0}
% Redefines (sub)paragraphs to behave more like sections
\ifx\paragraph\undefined\else
\let\oldparagraph\paragraph
\renewcommand{\paragraph}[1]{\oldparagraph{#1}\mbox{}}
\fi
\ifx\subparagraph\undefined\else
\let\oldsubparagraph\subparagraph
\renewcommand{\subparagraph}[1]{\oldsubparagraph{#1}\mbox{}}
\fi

\title{Using CriticMarkup with pandoc}
\author{Kolen Cheung}
\date{}

\begin{document}
\maketitle

Using CriticMarkup with pandoc---not a filter but a preprocessor.

\section{Caveats}\label{caveats}

The way this script works depends on the fact that pandoc allows RAW
HTML and RAW LaTeX in the markdown source. The CriticMarkup is
transformed into either a RAW HTML or RAW LaTeX representations
(specified by the command line arguments).

Because of the asymmetry in the way pandoc handle RAW HTML and RAW LaTeX
(namely, markdown inside RAW HTML are parsed, but not in RAW LaTeX),
markdown within the CriticMarkup will not be rendered in LaTeX output.
This filter might help if you want to change that:
\href{https://gist.github.com/mpickering/f1718fcdc4c56273ed52}{LaTeX
Argument Parser}.

\section{Definition of CriticMarkup}\label{definition-of-criticmarkup}

\begin{itemize}
\tightlist
\item
  Deletions: This is a test.
\item
  Additions: This is a test.
\item
  Substitutions: This is a test.
\item
  Highlighting: This is a test.
\item
  Comments: This is a test.
\end{itemize}

\section{The Scripts}\label{the-scripts}

These scripts are supposed to be in the same folder:

\begin{itemize}
\tightlist
\item
  \texttt{criticmarkup-accept.py}
\item
  \texttt{criticmarkup-reject.py}
\item
  \texttt{pandoc-criticmarkup.sh}
\end{itemize}

The \texttt{criticmarkup-accept.py} and \texttt{criticmarkup-reject.py}
are extracted from the OS X System Services from
\href{http://criticmarkup.com/services.php}{CriticMarkup Toolkit}.

\subsection{Note on LaTeX Output: Usepackage
Required}\label{note-on-latex-output-usepackage-required}

Note that the latex output requires the LaTeX packages \texttt{color}
and \texttt{soul}. As you can see from this markdown file, I used a hack
to guarantee your pandoc standard template also use them, like this:
\texttt{fontfamily:\ lmodern,color,soul}.

In my personal use, I have an YAML of
\texttt{usepackage:\ {[}color,soul{]}} that in my template will added
\texttt{\textbackslash{}usepackage\{color,soul\}}. See
\href{https://github.com/ickc/pandoc-templates/tree/latex-usepackage-hyperref}{ickc/pandoc-templates
at latex-usepackage-hyperref} for details.

\section{Usage}\label{usage}

\texttt{pandoc-criticmarkup.sh\ {[}options...{]}\ {[}file{]}}

Options:

\begin{itemize}
\tightlist
\item
  accept: \texttt{-a}
\item
  reject: \texttt{-r}
\item
  permanent: \texttt{-p}
\item
  show diff: \texttt{-d}

  \begin{itemize}
  \tightlist
  \item
    \texttt{-d\ html}: targeting html output using RAW HTML
  \item
    \texttt{-d\ latex}: targeting LaTeX output using RAW LaTeX
  \item
    \texttt{-d\ pdf}: same as above
  \end{itemize}
\end{itemize}

If permanent is used, it will overwrite the original, if not, it will
output to \texttt{stdout}. In most situation permanent should be used
with \texttt{-a} or \texttt{-r} only, but it can be used with
\texttt{-d} as well.

\texttt{-a}, \texttt{-r}, \texttt{-d} are supposed to use separately:

\begin{itemize}
\tightlist
\item
  If \texttt{-d} is used, the others are ignored,
\item
  if \texttt{-r} is used, \texttt{-a} is ignored
\end{itemize}

It can be used with the pandoc commands, like these (see
\texttt{build.sh} for more examples):

\begin{Shaded}
\begin{Highlighting}[]
\CommentTok{## Showing Difference and not overwriting}
\KeywordTok{./pandoc-criticmarkup.sh} \NormalTok{-d html README.md }\KeywordTok{|} \KeywordTok{pandoc} \NormalTok{-s -o README.html}
\KeywordTok{./pandoc-criticmarkup.sh} \NormalTok{-d pdf README.md }\KeywordTok{|} \KeywordTok{pandoc} \NormalTok{-s -o README.pdf}
\CommentTok{## accept or reject while overwriting the source}
\KeywordTok{./pandoc-criticmarkup.sh} \NormalTok{-ap README-accept.md}
\KeywordTok{./pandoc-criticmarkup.sh} \NormalTok{-rp README-reject.md}
\end{Highlighting}
\end{Shaded}

\section{Appendix}\label{appendix}

\subsection{CSS}\label{css}

An optional CSS \texttt{pandoc-criticmarkup.css} make the deletions and
additions more obvious in HTML output.

\subsection{Mapping for Showing
Differences}\label{mapping-for-showing-differences}

\begin{longtable}[]{@{}lll@{}}
\toprule
critic markup & HTML & LaTeX\tabularnewline
\midrule
\endhead
\bottomrule
\end{longtable}

`\texttt{\textbar{}}{[}text{]}\texttt{\textbar{}}\st{[text]}`
\textbar{}\\
\texttt{{[}text{]}} \textbar{}
\texttt{\textless{}ins\textgreater{}{[}text{]}\textless{}/ins\textgreater{}}
\textbar{} \texttt{\textbackslash{}underline\{{[}text{]}\}} \textbar{}\\
\texttt{{[}text2{]}} \textbar{}
\texttt{\textless{}del\textgreater{}{[}text1{]}\textless{}/del\textgreater{}\textless{}ins\textgreater{}{[}text2{]}\textless{}/ins\textgreater{}}
\textbar{}
\texttt{\textbackslash{}st\{{[}text1{]}\}\textbackslash{}underline\{{[}text2{]}\}}
\textbar{}\\
\texttt{{[}text{]}} \textbar{}
\texttt{\textless{}mark\textgreater{}{[}text{]}\textless{}/mark\textgreater{}}
\textbar{} \texttt{\textbackslash{}hl\{{[}text{]}\}} \textbar{}\\
\texttt{\textless{}!-\/-\ {[}text{]}\ -\/-\textgreater{}} \textbar{}
\texttt{\textless{}aside\textgreater{}{[}text{]}\textless{}/aside\textgreater{}}
\textbar{} \texttt{\textbackslash{}marginpar\{{[}text{]}\}} \textbar{}

\end{document}
